\documentclass{article}
\usepackage{amsmath}
\usepackage{cases}
\usepackage{amsfonts}
\usepackage [a4paper, left=1cm, right=2cm, top=2cm, bottom=4cm]{geometry}
\usepackage{hyperref}
\usepackage{titlesec}
\usepackage{graphicx}

\title{Accompaniment Generation}
\author{Alexandr Ragulin}

% \documentclass[11pt, twocolumn]{article}

% \graphicspath{ {./img/} }

\hypersetup{
  colorlinks=true,
  linkcolor=blue,
  filecolor=magenta,
  urlcolor=blue,
  pdfpagemode=FullScreen,
}



\begin{document}
\maketitle

\section{Running the program}
Full project is uploaded to the github repository:
\href{https://github.com/Skril3366/accompaniment-generation.hw}{https://github.com/Skril3366/accompaniment-generation.hw}.
The process of running program is described in the README.md file and
dependencies are specified in the requirements.txt file.

\subsection{Requirements}

\begin{itemize}
  \item Python 3.10.8 or higher
  \item pygame 2.1.2
  \item mido 1.2.10
  \item music21 8.1.0
  \item argparse 1.4.0
\end{itemize}

\subsection{Running the program}

The only argument required is path to the midi file. Here is sample command:
\begin{verbatim}
python3 AlexandrRagulin.py ./resources/babiegirl_mono.mid
\end{verbatim}

\section{Detected keys}

\begin{itemize}
  \item barbiegirl\_mono.mid: C\#m
  \item input1.mid: Dm
  \item input2.mid: F
  \item input3.mid: Em
\end{itemize}

\section{Key detection algorithm}

For detection key of the song I used Krumhansl-Shmuckler algorithm. It uses
statistical analysis of the relative frequencies of the notes.

I've attempted to implement it, however, I didn't manage to find a bug in my
implementation, so I decided to stick to one provided by music21 library.

\section{Genetic algorithm}

My implementation of genetic algorithm generates new population by applying the
following procedured to the previous one:
\begin{itemize}
  \item Crossover
  \item Mutation
\end{itemize}

Firstly, an "elite"(ones with the best fitness function) is selected from the
population, then the crossover is performed on the best ones(number of elites
and number of individuals to crossover may not be the same). Then this two lists
of individuals are merged and mutation is performed on them.

\subsection{Fitness function}

Fitness function consists of 2 parts:
\begin{itemize}
  \item Vertical fitness
  \item Horizontal fitness
\end{itemize}

Vertical fitness is calculated by examining how much dissonant notes are played
at the same time.

Horizontal fitness is calculated by:
\begin{itemize}
  \item Number of dissonant notes played one after another
  \item Correctness of harmonic movement of the chords
  \item Penalty for repeating chords (the closer repeated chords, the worse)
\end{itemize}

Harmonic movement (from one chord to another) is prioritized if it is:
\begin{itemize}
  \item Dominant $\rightarrow$ tonic
  \item Subdominant $\rightarrow$ dominant or tonic
  \item Tonic $\rightarrow$ subdominant or dominant
\end{itemize}



\end{document}
